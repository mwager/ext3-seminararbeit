
\begin{abstract}

Dieses Dokument befasst sich mit dem Linux-Dateisystem EXT3 und wurde im Rahmen einer Seminararbeit für Masterstudierende im Fach \textit{Einführung in die IT Forensik} erstellt. Es ist grob in zwei Teile aufgeteilt: Im ersten Teil wird einleitend auf die Geschichte von Dateisystemen unter Linux eingegangen und anschließend die grundlegenden Datenstrukturen sowie die Funktionsweise des EXT3 Dateisystems beschrieben. Im zweiten Teil beginnt dann die technische Analyse, welche forensische Konzepte zur Auswertung des Dateisystems, sowie die Wiederherstellung gelöschter Dateien untersucht. Zur Veranschaulichung der technischen Hintergründe beinhalten beide Teile praktische Beispiele, welche der Leser auf einem gängigen Linux-System wie z.B. Ubuntu einfach nachvollziehen kann.

\end{abstract}
